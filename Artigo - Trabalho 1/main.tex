\documentclass[conference]{IEEEtran}
\IEEEoverridecommandlockouts
% The preceding line is only needed to identify funding in the first footnote. If that is unneeded, please comment it out.
\usepackage{cite}
\usepackage{amsmath,amssymb,amsfonts}
\usepackage{algorithmic}
\usepackage{graphicx}
\usepackage{textcomp}
\usepackage{xcolor}
\def\BibTeX{{\rm B\kern-.05em{\sc i\kern-.025em b}\kern-.08em
    T\kern-.1667em\lower.7ex\hbox{E}\kern-.125emX}}
\begin{document}

\title{Projeto (\textit{design}) de Software*\\
}

\author{\IEEEauthorblockN{1\textsuperscript{st} Mariana Borges de Sampaio}
\textit{Matrícula: 180046926}\\
\textit{Brasília, Brasil} \\
\textit{180046926@aluno.unb.br}}



\maketitle

\begin{abstract}Due to the constant growth of technology over the years, some studies are done with the aim of improving, making software development more efficient. For that, it is necessary to have prior knowledge in order to be successful in the product or service that will be developed. With this in mind for better learning, this article will demonstrate the definition and the knowledge necessary for a software project to be developed with full knowledge and foundation before developing a product. Since before a product is developed, it is necessary to build a software project.

\end{abstract}

\begin{IEEEkeywords}
Software design, software design, product, service,
software architecture

\end{IEEEkeywords}

\section{resumo} Devido ao constante crescimento da tecnologia ao decorrer dos anos, alguns estudos são feitos com intuito de melhorar, de tornar mais eficiente o desenvolvimento de software. Para isso é necessário que se tenha um conhecimento prévio a fim de obter êxito no produto ou serviço que será desenvolvido. Tendo isso em vista um melhor aprendizado, este artigo demonstrará a definição e os conhecimentos necessários para que seja desenvolvido com total conhecimento e embasamento um projeto de software antes de desenvolver um produto. Uma vez que antes de ser desenvolvido um produto é necessário que seja construído um projeto de software.

\begin{IEEEkeywords}
Projeto de software, design de software, produto, serviço,arquitetura de software
\end{IEEEkeywords}

\section{Introdução}
Neste artigo o termo de projeto de software terá referência ao design de software. Isso ocorre pois ao projetar uma aplicação o que é é feito  é chamado de design de software ou projeto de software. Ou seja, toda a estrutura a ser desenvolvida para construir a arquitetura do software, a definição de como os dados serão armazenados,como o banco de dados será construído , a definição de qual IDE será necessário, como será feita a conexão entre o banco de dados e a sua aplicação, quais interações devem ser feitas entre os objetos entre demais atividades que podem ser definidas elas compõem o design de software. Como definição de design de software a ser seguida, tem-se a seguinte: 
  
    "Design is defined as both “the process of defin-
    ing the architecture, components, interfaces, and
    other characteristics of a system or component”
    and “the result of [that] process”."~\cite{SWEBOK2014}

\section{Desenvolvimento}

\section{Conclusão}



{\small
\bibliography{egbib}
\bibliographystyle{ieee_fullname}

}

\end{document}