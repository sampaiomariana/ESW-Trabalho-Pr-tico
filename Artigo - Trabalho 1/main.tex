\documentclass[conference]{IEEEtran}
\IEEEoverridecommandlockouts
% The preceding line is only needed to identify funding in the first footnote. If that is unneeded, please comment it out.
\usepackage{cite}
\usepackage{amsmath,amssymb,amsfonts}
\usepackage{algorithmic}
\usepackage{graphicx}
\usepackage{textcomp}
\usepackage{xcolor}
\def\BibTeX{{\rm B\kern-.05em{\sc i\kern-.025em b}\kern-.08em
    T\kern-.1667em\lower.7ex\hbox{E}\kern-.125emX}}
\begin{document}

\title{Projeto (\textit{design}) de Software*\\
}

\author{\IEEEauthorblockN{1\textsuperscript{st} Mariana Borges de Sampaio}
\textit{Matrícula: 180046926}\\
\textit{Brasília, Brasil} \\
\textit{180046926@aluno.unb.br}}



\maketitle

\begin{abstract}Due to the constant growth of technology over the years, some studies are done with the aim of improving, making software development more efficient. For that, it is necessary to have prior knowledge in order to be successful in the product or service that will be developed. With this in mind for better learning, this article will demonstrate the definition and the knowledge necessary for a software project to be developed with full knowledge and foundation before developing a product. Since before a product is developed, it is necessary to build a software project.

\end{abstract}

\begin{IEEEkeywords}
Software design, software design, product, service,
software architecture

\end{IEEEkeywords}

\section{resumo} Devido ao constante crescimento da tecnologia ao decorrer dos anos, alguns estudos são feitos com intuito de melhorar, de tornar mais eficiente o desenvolvimento de software. Para isso é necessário que se tenha um conhecimento prévio a fim de obter êxito no produto ou serviço que será desenvolvido. Tendo isso em vista um melhor aprendizado, este artigo demonstrará a definição e os conhecimentos necessários para que seja desenvolvido com total conhecimento e embasamento um projeto de software antes de desenvolver um produto. Uma vez que antes de ser desenvolvido um produto é necessário que seja construído um projeto de software.

\begin{IEEEkeywords}
Projeto de software, design de software, produto, serviço,arquitetura de software
\end{IEEEkeywords}

\section{Introdução}
Neste artigo o termo de projeto de software terá referência ao design de software. Isso ocorre pois ao projetar uma aplicação o que é é feito  é chamado de design de software ou projeto de software. Ou seja, toda a estrutura a ser desenvolvida para construir a arquitetura do software, a definição de como os dados serão armazenados,como o banco de dados será construído , a definição de qual IDE será necessário, como será feita a conexão entre o banco de dados e a sua aplicação, quais interações devem ser feitas entre os objetos entre demais atividades que podem ser definidas elas compõem o design de software. Como definição de design de software a ser seguida, tem-se a seguinte: 
  
"Design is defined as both “the process of defining the architecture, components, interfaces, and other characteristics of a system or component” and “the result of [that] process”.
    "~\cite{SWEBOK2014}
    
Analisando essa definição, nota-se que o design de projeto possui como principais atividades desenvolver a arquitetura do sistema de software, definir quais componentes serão utilizados, definição da interface do sistema, entre outras atividades que terão como centro elementos que fazem parte do sistema que está sendo construído.Sendo assim, pode-se observar que o design de software possui duas bases a serem seguidas. Primeiramente, tem-se o desenvolvimento da arquitetura do software e posteriormente tem-se a estruturação de como será feito o design do projeto.Sendo estes dois elementos os processos definidos pelo design de software.

"• Architectural design (also referred to as high-
level design and top-level design) describes
how software is organized into components."~\cite{SWEBOK2014}

"• Detailed design describes the desired behav-
ior of these components." ~\cite{SWEBOK2014}

Por meio de estudos desenvolvidos na área de Engenharia de Software pode-se perceber que para construir um sistema de software é necessária uma engenharia por trás do seu processo. Tem-se como fases de sistema de construção do software:

• Gerenciamento de Projeto

• Projeto de software

• Gerenciamento de Configuração

• Modelagem de software

• Processos de software

• Requisitos de software

• Design de software

• Qualidade de software

• Manutenção e reúso

Diante desses temas, este artigo trata diretamente do tema de Design de software que abordará o desenvolvimento da arquitetura do software que é necessária para o desenvolvimento de um sistema. 

\section{Desenvolvimento}

A arquitetura de um sistema de software tem como principal função ser a estrutura do sistema, ou seja, é a arquitetura do sistema que demonstra como ele funciona independente de qual seja o nível de granularidade do sistema. Existem alguns processos de arquitetura que podem ser escolhidos para fazer parte de um determinado sistema, essa escolha deve ser pensada de acordo com o funcionamento e como deve ser feito o nível de iteratividade do usuário com o sistema e do desenvolvedor com o sistema. 
Como a arquitetura do sistema engloba o sistema como um todo, ela não precisa ser construída em sua versão final logo no início do projeto. Isso ocorre pois  a arquitetura deve mostrar como o sistema funciona, tendo dessa forma que indicar qual deve ser o primeiro e útlimo passo do usuário com o software.Por isso a arquitetura pode oscilar muito em relação a como ela é pensada no início do projeto a como ela de fato será implementada. 
\section{Conclusão}



\bibliographystyle{plain}
\bibliography{egbib}


\end{document}